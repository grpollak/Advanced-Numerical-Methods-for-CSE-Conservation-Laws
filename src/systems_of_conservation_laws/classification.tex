\begin{defnbox}\nospacing
    \begin{defn}[Hyperbolic System]\label{defn:hyperbolic_system}\leavevmode\\
        The linear system\cref{eq:linear_system_of_conservation_law,eq:linear_system_of_conservation_law_variable_coef}
        are called \textit{hyperbolic} if the matrix $\Avec$ is diagonalizable and has $\idxm$ real eigenvalues:
        \begin{align}
          \text{spectrum}(\Avec)(\xvec,t)=\left\{\eigenval(\xvec,t)_{1},\ldots,\eigenval(\xvec,t)_{\idxm}\right\}\in\R&&\forall \xvec,t\label{eq:hyperbolic_system}
        \end{align}
    \end{defn}
\end{defnbox}
\begin{corbox}\nospacing
    \begin{cor}[Strictly Hyperbolic System]\label{cor:strictly_hyperbolic_system}\leavevmode\\
        The linear system\cref{eq:linear_system_of_conservation_law,eq:linear_system_of_conservation_law_variable_coef}
        is called \textit{strictly hyperbolic} if it is \textit{hyperbolic}\cref{defn:hyperbolic_system} and all eigenvalues are distinct:
        \begin{align}
          \cref{eq:hyperbolic_system}&&+&&\eigenval_{1}\neq\eigenval_{2}\neq\ldots\neq\eigenval_{\idxm}
        \end{align}
    \end{cor}
\end{corbox}

%%% Local Variables:
%%% mode: latex
%%% TeX-command-extra-options: "-shell-escape"
%%% TeX-master: "../../formulary"
%%% End:
