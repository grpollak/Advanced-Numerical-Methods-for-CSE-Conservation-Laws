\begin{defnbox}\nospacing
    \begin{defn}[\hfill\proofref{proof:defn:roe_matrices}\newline Roe Matrices\hfill\tcblack{$\linfluxvec_{\idxj+1/2}^{n}$}]\label{defn:roe_matrices}\leavevmode\\
        Are matrices that satisfy the \cref{property:strict_hyperbolicity,property:consistency,property:,property:roes_criterion}
        \begin{align}
          \linfluxvec_{\idxj+1/2}^{n}=\int_{0}^{1}\fluxvec' \left(\uvec_{\idxj}^{n}+\tauc \left(\uvec_{\idxj+1}^{n}-\uvec_{\idxj}^{n}\right)\right)\diff\tauc\label{eq:roe_matrices}
        \end{align}
    \end{defn}
\end{defnbox}
\begin{sectionbox}[Problem]\nospacing
   \Cref{eq:roe_matrices} is not easy to calculate and in general not possible to calculate in general.
\end{sectionbox}
\begin{propositionbox}\nospacing
    \begin{proposition}[\hfill\exampleref{example:proposition:roe_matrix_shallow_water,example:proposition:roe_matrix_euler}\newline Roe Matrix]\leavevmode\\
        \label{proposition:roe_matrix}
        We derive the row matrix by solving \cref{eq:roes_criterion_non_linear_systems}:\\
        \scalematho[0.98]{\begin{align*}
          &[[\fluxvec]]=\linfluxvec[[\uvec]]\iff
          \fluxvec\left(\uvec_{\idxj+1}^{n}\right)-\fluxvec\left(\uvec_{\idxj}^{n}\right)=\linfluxvec_{\idxj+1/2}^{n}\left(\uvec_{\idxj+1}^{n}-\uvec_{\idxj}^{n}\right)
        \end{align*}}
        using a clever change of variables depenindg on the underlying problem:
        \begin{align}
          &\Zvec:\Uvec\mapsto\Zvec(U)&&\Zvec\in\admissibleset\subset\R^{\idxm}
        \end{align}
    \end{proposition}
\end{propositionbox}
\begin{explanationbox}\nospacing
    \begin{explanation}
        When writing down \cref{eq:roes_criterion_non_linear_systems} we often arrive at rational equations.
        By a cleaver change of variables we can transform those equations into polynomial equations, which are much easier to solve.
    \end{explanation}
\end{explanationbox}
\todo[inline]{Why/Does this automatically satisfy all the equations?}
\begin{formulabox}\nospacing
   \begin{formula}[Useful Identities]
       \begin{align}
         \widebar{a}:=\frac{a_{l}+a_{r}}{2}&&\dbracket{a}:=a_{r}-a_{l}
       \end{align}
       \begin{align}
         &\dbracket{ab}=\widebar{b}\dbracket{a}+\widebar{a}\dbracket{b} \\[-1\jot]
         &\dbracket{a^{2}}=2\widebar{a}\dbracket{a} \\[-1\jot]
         &\dbracket{a^{4}}=4\widebar{a^{2}}\widebar{a}\dbracket{a}
       \end{align}
   \end{formula}
\end{formulabox}

%%% Local Variables:
%%% mode: latex
%%% TeX-command-extra-options: "-shell-escape"
%%% TeX-master: "../../../../formulary"
%%% End:
