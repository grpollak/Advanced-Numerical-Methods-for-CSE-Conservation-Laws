  \begin{propertybox}\nospacing
      \begin{property}[Strict Hyperbolicity]\label{property:strict_hyperbolicity}\leavevmode\\
          $\linfluxvec_{\idxj+1/2}^{n}\in\R^{\idxm\times\idxm}$ should be strictly hyperbolic\cref{cor:strictly_hyperbolic_system}.
      \end{property}
  \end{propertybox}
  \begin{propertybox}\nospacing
      \begin{property}[Consistency]\label{property:consistency}\leavevmode\\
          $\linfluxvec_{\idxj+1/2}^{n}=\linfluxvec_{\idxj+1/2}^{n}\left(\uvec_{\idxj}^{n},\uvec_{\idxj}^{n+1}\right)$ should be consistent:
          \begin{align}
            \linfluxvec_{\idxj+1/2}^{n}\left(\uvec,\uvec\right)=\fluxvec'(\uvec)\label{eq:property:consistency}
          \end{align}
      \end{property}
  \end{propertybox}
\begin{explanationbox}\nospacing
    \begin{explanation}
        If the left and right states are consistent/have the same value then our approximation should do nothing and be equal to the real flux.
    \end{explanation}
\end{explanationbox}
  \begin{propertybox}\nospacing
      \begin{property}[\hfill\proofref{proof:property:roes_criterion}\newline Roes Criterion]\label{property:roes_criterion}
          Isolated Discontinuities should be preserved exactly by our approximation:
          \begin{align}
            \fluxvec\left(\uvec_{\idxj+1}^{n}\right)-\fluxvec\left(\uvec_{\idxj}^{n}\right)=\linfluxvec_{\idxj+1/2}^{n}\left(\uvec_{\idxj+1}^{n}-\uvec_{\idxj}^{n}\right)
            \label{eq:roes_criterion_non_linear_systems}
          \end{align}
      \end{property}
  \end{propertybox}

%%% Local Variables:
%%% mode: latex
%%% TeX-command-extra-options: "-shell-escape"
%%% TeX-master: "../../../../formulary"
%%% End:
