\begin{defnbox}\nospacing
    \begin{defn}[\newline HLL original]\label{defn:hll_original}\leavevmode\\
        Approximates the wave cone to capture everything:
        \begin{align}
          s_{\idxj+1/2}^{l,n}=\min\left(\eigenval_{\idxj}^{\idxp[1],n},\eigenval_{\idxj+1}^{\idxp[1],n}\right)&&
          s_{\idxj+1/2}^{r,n}=\max\left(\eigenval_{\idxj}^{\idxm,n},\eigenval_{\idxj+1}^{\idxm,n}\right)
        \end{align}
    \end{defn}
\end{defnbox}
\begin{sectionbox}\nospacing
    \begin{minipage}[t]{0.4\textwidth}
       \begin{proslist}
           \item Takes into account local information
           \item No longer symmetric, thus can capture unidirectional wavs
       \end{proslist}
    \end{minipage}
    \begin{minipage}[t]{0.55\textwidth}
       \begin{conslist}
           \item Is still an approximation consisting just of three waves i.e.\ already for
           three waves it will no longer model the middle wave.
       \end{conslist}
    \end{minipage}
\end{sectionbox}
%%% Local Variables:
%%% mode: latex
%%% TeX-command-extra-options: "-shell-escape"
%%% TeX-master: "../../../../../../formulary"
%%% End:
