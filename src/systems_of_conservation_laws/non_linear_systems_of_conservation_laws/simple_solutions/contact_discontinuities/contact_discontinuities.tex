\begin{lemmabox}\nospacing
    \begin{lemma}[Existence Contact Discontinuity]\label{lemma:existence_contact_discontinuity}\leavevmode\\
        Let the $\idxj$-th wave family\cref{defn:wave_family} be \textit{linear degenerate}\cref{defn:linearly_degenerat_wave_family}
        and let $\Uvec_{L}\in\admissibleset$.
        Then by the Piccard-Lindeloef Theorem\cref{defn:todo} there exists an \textit{integral curve} solving \cref{eq:simple_ode}:
        \begin{align}
          \mathcal{C}_{\idxj}\left(\Uvec_{L}\right)=\left\{\Wvec_{\idxj}\left(\epsilonc^{*}\right)\in\R^{\idxm}: \epsilonc^{*}\in[-\widebar{\epsilonc},\widebar{\epsilonc})\right\}
        \end{align}
        if $\Uvec_{R}\in\mathcal{C}_{\idxj}\left(\Uvec_{L}\right)$ then there exists a \textit{contact discontinuity solution}\cref{defn:contact_discontinuity_solution_non_linear_systems_of_equations} $\Uvec$ to the Riemann problem \cref{eq:riemann_problem_nonlinear_systems_of_conservation_laws}.
    \end{lemma}
\end{lemmabox}
\begin{defnbox}\nospacing
    \begin{defn}[\hfill\proofref{proof:contact_discontinuity}
        \newline Contact Discontinuity Solution]
        \label{defn:contact_discontinuity_solution_non_linear_systems_of_equations}\leavevmode\\
        If \cref{lemma:existence_contact_discontinuity} is satisfied then the solution of \cref{eq:riemann_problem_nonlinear_systems_of_conservation_laws} is given by:\\
        \begin{minipage}{0.55\textwidth}
        \begin{align}
          &\Uvec(x,t)=
          \left\{\begin{aligned}
              &\Uvec_{L}&&\frac{x}{t}<\eigenval_{\idxj}\left(\Uvec_{L}\right) \\
              &\Uvec_{R}&&\eigenval_{\idxj}\left(\Uvec_{R}\right)<\frac{x}{t}
          \end{aligned}\right.\label{eq:contact_discontinuity_solution_non_linear_systems_of_equations}
        \end{align}\hfill
        \end{minipage}
        \begin{minipage}{0.44\textwidth}
        \begin{figure}[H]
            \centering{
              \def\svgwidth{100pt}
              \resizebox{\linewidth}{!}{
                \input{src/systems_of_conservation_laws/non_linear_systems_of_conservation_laws/simple_solutions/figures/characteristics_contact_discontinuity.pdf_tex}}
            }
        \end{figure}
        \end{minipage}
    \end{defn}
\end{defnbox}
\begin{explanationbox}\nospacing
    \begin{explanation}
        Appear in gas genomics when a with a discontinuity in mass density but not in the pressure or velocity, in comparison
        to real shocks, which move faster than the gas itself due to a discontinuity in pressure.
    \end{explanation}
\end{explanationbox}
\begin{defnbox}\nospacing
    \begin{defn}[\hfill\proofref{proof:defn:rankine-hugoniot_condition_contact_discontinuity}\newline Rankine-Hugoniot Condition]\label{defn:rankine-hugoniot_condition_contact_discontinuity}\leavevmode\\
        A contact discontinuity solution\cref{defn:contact_discontinuity_solution_non_linear_systems_of_equations}
        fulfills the Rankine-Hugoniot Condition:
        \begin{align}
          \flux\left(\Uvec_{R}\right)-\flux\left(\Uvec_{L}\right)
          =s \left(\Uvec_{R}-\Uvec_{L}\right)&&s:=\eigenval_{\idxj} \left(\Uvec_{R}\right)=\eigenval_{\idxj} \left(\Uvec_{L}\right)
        \end{align}
    \end{defn}
\end{defnbox}


%%% Local Variables:
%%% mode: latex
%%% TeX-master: "../../../../../formulary"
%%% TeX-command-extra-options: "-shell-escape"
%%% End:
