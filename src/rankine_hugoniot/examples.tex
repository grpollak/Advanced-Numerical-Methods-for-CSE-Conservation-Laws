\begin{examplebox}\nospacing
    \begin{example}[RK for Riemann Problem\cref{defn:rankine-hugoniot_condition}]\label{example:rk_for_riemann_problem}
        \leavevmode\\
        \begin{minipage}{0.4\textwidth}
            \begin{align}
            u_{t}+\left(\frac{u^{2}}{2}\right)_{x}=0
            \end{align}
            \begin{align}
            u_{0}=\begin{cases}
                    1&\text{if }x<0\\
                    0&\text{if }x>0
                    \end{cases}
            \label{eq:riemann_problem}
            \end{align}
        \end{minipage}\hfil
        \begin{minipage}[c]{0.4\textwidth}
            \begin{figure}[H]
                \centering{
                  \def\svgwidth{100pt}
                  \resizebox{\linewidth}{!}{\begin{defnbox}\nospacing
    \begin{defn}[Riemann Problem]\label{defn:riemann_problem}
        Is an initial value problem of a conservation law with
        picewise initial data with a single discontinuity of the form:\\
        \begin{align}
        u_{t}+\flux \left(u\right)_{x}=0&&
        u_{0}=\begin{cases}
                U_{R}&\text{if }x>0\\
                U_{L}&\text{if }x<0
                \end{cases}
        \end{align}
        \begin{minipage}{0.49\textwidth}
            \begin{figure}[H]
                \centering{
                  \def\svgwidth{100pt}
                  \resizebox{\linewidth}{!}{\input{src/conservation_laws/riemann_problem/figures/riemann_problem_shock.pdf_tex}}
                }
                \caption{$U_{L}<U_{R}$}
            \end{figure}
        \end{minipage}\hfil
        \begin{minipage}[c]{0.49\textwidth}
            \begin{figure}[H]
                \centering{
                  \def\svgwidth{100pt}
                  \resizebox{\linewidth}{!}{\begin{defnbox}\nospacing
    \begin{defn}[Riemann Problem]\label{defn:riemann_problem}
        Is an initial value problem of a conservation law with
        picewise initial data with a single discontinuity of the form:\\
        \begin{align}
        u_{t}+\flux \left(u\right)_{x}=0&&
        u_{0}=\begin{cases}
                U_{R}&\text{if }x>0\\
                U_{L}&\text{if }x<0
                \end{cases}
        \end{align}
        \begin{minipage}{0.49\textwidth}
            \begin{figure}[H]
                \centering{
                  \def\svgwidth{100pt}
                  \resizebox{\linewidth}{!}{\input{src/conservation_laws/riemann_problem/figures/riemann_problem_shock.pdf_tex}}
                }
                \caption{$U_{L}<U_{R}$}
            \end{figure}
        \end{minipage}\hfil
        \begin{minipage}[c]{0.49\textwidth}
            \begin{figure}[H]
                \centering{
                  \def\svgwidth{100pt}
                  \resizebox{\linewidth}{!}{\begin{defnbox}\nospacing
    \begin{defn}[Riemann Problem]\label{defn:riemann_problem}
        Is an initial value problem of a conservation law with
        picewise initial data with a single discontinuity of the form:\\
        \begin{align}
        u_{t}+\flux \left(u\right)_{x}=0&&
        u_{0}=\begin{cases}
                U_{R}&\text{if }x>0\\
                U_{L}&\text{if }x<0
                \end{cases}
        \end{align}
        \begin{minipage}{0.49\textwidth}
            \begin{figure}[H]
                \centering{
                  \def\svgwidth{100pt}
                  \resizebox{\linewidth}{!}{\input{src/conservation_laws/riemann_problem/figures/riemann_problem_shock.pdf_tex}}
                }
                \caption{$U_{L}<U_{R}$}
            \end{figure}
        \end{minipage}\hfil
        \begin{minipage}[c]{0.49\textwidth}
            \begin{figure}[H]
                \centering{
                  \def\svgwidth{100pt}
                  \resizebox{\linewidth}{!}{\input{src/conservation_laws/riemann_problem/figures/riemann_problem.pdf_tex}}
                }
                \caption{$U_{L}>U_{R}$}
            \end{figure}
        \end{minipage}
    \end{defn}
\end{defnbox}

%%% Local Variables:
%%% mode: latex
%%% TeX-command-extra-options: "-shell-escape"
%%% TeX-master: "../../../formulary"
%%% End:
}
                }
                \caption{$U_{L}>U_{R}$}
            \end{figure}
        \end{minipage}
    \end{defn}
\end{defnbox}

%%% Local Variables:
%%% mode: latex
%%% TeX-command-extra-options: "-shell-escape"
%%% TeX-master: "../../../formulary"
%%% End:
}
                }
                \caption{$U_{L}>U_{R}$}
            \end{figure}
        \end{minipage}
    \end{defn}
\end{defnbox}

%%% Local Variables:
%%% mode: latex
%%% TeX-command-extra-options: "-shell-escape"
%%% TeX-master: "../../../formulary"
%%% End:
}
                }
            \end{figure}
        \end{minipage}
        \begin{align*}
          s(t)=\sigma'(t)&=\frac{\flux \left(u^{-}(t)\right)-\flux \left(u^{+}(t)\right)}{
          u^{-}(t)-u^{+}(t)
          }=\frac{\flux(1)-\flux(0)}{1-0}\\[-1\jot]
          &=\frac{\frac{1}{2}-0}{1}=\frac{1}{2}\quad\Rightarrow\quad\sigma(t)=\frac{t}{2}
        \end{align*}
        \begin{minipage}[c]{0.55\textwidth}
            \begin{align}
            u=\begin{cases}
                    1&\text{if }x<\frac{t}{2}\\
                    0&\text{if }x>\frac{t}{2}
                    \end{cases}
            \label{eq:riemann_problem}
            \end{align}
        \end{minipage}
        \vspace{-1em}
        \begin{minipage}{0.3\textwidth}
            \begin{figure}[H]
                \centering{
                  \def\svgwidth{50pt}
                  \resizebox{0.9\linewidth}{!}{\input{src/rankine_hugoniot/figures/riemann_problem_speed.pdf_tex}}
                }
            \end{figure}
        \end{minipage}
        \begin{minipage}{0.4\textwidth}
            \begin{figure}[H]
                \centering{
                  \def\svgwidth{70pt}
                  \resizebox{\linewidth}{!}{\input{src/rankine_hugoniot/figures/riemann_problem_char.pdf_tex}}
                }
            \end{figure}
        \end{minipage}\hfill
        \begin{minipage}{0.55\textwidth}
            \begin{figure}[H]
                \centering{
                  \def\svgwidth{110pt}
                  \resizebox{\linewidth}{!}{\input{src/rankine_hugoniot/figures/riemann_problem_moving.pdf_tex}}
                }
            \end{figure}
        \end{minipage}
        Thus we found a weak solution, where the characteristics are colliding on a traveling discontinuity/shockwave\cref{defn:shock_waves}.
    \end{example}
\end{examplebox}
\begin{examplebox}\nospacing
    \begin{example}[RK for Riemann Problem emanating]\label{example:rk_for_riemann_problem_rarefaction}
        \begin{minipage}{0.4\textwidth}
            \begin{align}
            u_{t}+\left(\frac{u^{2}}{2}\right)_{x}=0
            \end{align}
            \begin{align}
            u_{0}=\begin{cases}
                    0&\text{if }x<0\\
                    1&\text{if }x>0
                    \end{cases}
            \label{eq:riemann_problem}
            \end{align}
        \end{minipage}\hfil
        \begin{minipage}[c]{0.4\textwidth}
            \begin{figure}[H]
                \centering{
                  \def\svgwidth{100pt}
                  \resizebox{\linewidth}{!}{\input{src/rankine_hugoniot/figures/riemann_problem_rarefaction.pdf_tex}}
                }
            \end{figure}
        \end{minipage}
        \begin{align*}
          s(t)=\sigma'(t)&=\frac{\flux \left(u^{-}(t)\right)-\flux \left(u^{+}(t)\right)}{
          u^{-}(t)-u^{+}(t)
          }=\frac{\flux(0)-\flux(1)}{0-1}\\[-1\jot]
          &=\frac{-\frac{1}{2}-0}{-1}=\frac{1}{2}\quad\Rightarrow\quad\sigma(t)=\frac{t}{2}
        \end{align*}
        \begin{minipage}[c]{0.55\textwidth}
            \begin{align}
            u(x,t)=\begin{cases}
                    0&\text{if }x<\frac{t}{2}\\
                    1&\text{if }x>\frac{t}{2}
                    \end{cases}
            \label{eq:riemann_problem}
            \end{align}
        \end{minipage}
        \vspace{-1em}
        \begin{minipage}{0.3\textwidth}
            \begin{figure}[H]
                \centering{
                  \def\svgwidth{50pt}
                  \resizebox{0.9\linewidth}{!}{\input{src/rankine_hugoniot/figures/riemann_problem_speed.pdf_tex}}
                }
            \end{figure}
        \end{minipage}
        \begin{minipage}{0.4\textwidth}
            \begin{figure}[H]
                \centering{
                  \def\svgwidth{70pt}
                  \resizebox{\linewidth}{!}{\input{src/rankine_hugoniot/figures/riemann_problem_char_rarefaction.pdf_tex}}
                }
            \end{figure}
        \end{minipage}\hfill
        \begin{minipage}{0.55\textwidth}
            \begin{figure}[H]
                \centering{
                  \def\svgwidth{110pt}
                  \resizebox{\linewidth}{!}{\input{src/rankine_hugoniot/figures/riemann_problem_moving_rarefaction.pdf_tex}}
                }
            \end{figure}
        \end{minipage}
        \imp{Problem} we now get an area with characteristics emanating from the shock, thus we cannot track them back to the initial data.\\
        This region of outflowing characteristics may in fact be filled in several ways see\cref{example:rk_for_riemann_problem_rarefaction_two}
    \end{example}
\end{examplebox}
\begin{examplebox}\nospacing
    \begin{example}[RK for Riemann Problem emanating]\label{example:rk_for_riemann_problem_rarefaction_two}
        \begin{minipage}{0.4\textwidth}
            \begin{align}
            u_{t}+\left(\frac{u^{2}}{2}\right)_{x}=0
            \end{align}
            \begin{align}
            u_{0}=\left\{\begin{aligned}
                    0\\
                    \frac{1}{2}\\
                    1
                    \end{aligned}\right.&&\text{if}&&
                    \begin{aligned}
                    &x<\frac{1}{4}t\\
                    \frac{1}{4}t <&x<\frac{3}{4}t\\
                    &x>\frac{3}{4}t
                    \end{aligned}
            \label{eq:riemann_problem}
            \end{align}
        \end{minipage}\hfil
        \begin{minipage}[c]{0.4\textwidth}
            \begin{figure}[H]
                \centering{
                  \def\svgwidth{100pt}
                  \resizebox{\linewidth}{!}{\input{src/rankine_hugoniot/figures/riemann_problem_rarefaction.pdf_tex}}
                }
            \end{figure}
        \end{minipage}
        \begin{align*}
          s(t)=\sigma'(t)&=\frac{\flux \left(u^{-}(t)\right)-\flux \left(u^{+}(t)\right)}{
          u^{-}(t)-u^{+}(t)
          }=\frac{\flux(0)-\flux(1)}{0-1}\\[-1\jot]
          &=\frac{-\frac{1}{2}-0}{-1}=\frac{1}{2}\quad\Rightarrow\quad\sigma(t)=\frac{t}{2}
        \end{align*}
        \begin{minipage}[c]{0.55\textwidth}
            \begin{align}
            u(x,t)=\begin{cases}
                    0&\text{if }x<\frac{t}{2}\\
                    1&\text{if }x>\frac{t}{2}
                    \end{cases}
            \label{eq:riemann_problem}
            \end{align}
        \end{minipage}
        \vspace{-1em}
        \begin{minipage}{0.3\textwidth}
            \begin{figure}[H]
                \centering{
                  \def\svgwidth{50pt}
                  \resizebox{0.9\linewidth}{!}{\input{src/rankine_hugoniot/figures/riemann_problem_speed.pdf_tex}}
                }
            \end{figure}
        \end{minipage}
        \begin{minipage}{0.4\textwidth}
            \begin{figure}[H]
                \centering{
                  \def\svgwidth{70pt}
                  \resizebox{\linewidth}{!}{\input{src/rankine_hugoniot/figures/riemann_problem_char_rarefaction.pdf_tex}}
                }
            \end{figure}
        \end{minipage}\hfill
        \begin{minipage}{0.55\textwidth}
            \begin{figure}[H]
                \centering{
                  \def\svgwidth{110pt}
                  \resizebox{\linewidth}{!}{\input{src/rankine_hugoniot/figures/riemann_problem_moving_rarefaction.pdf_tex}}
                }
            \end{figure}
        \end{minipage}
        This solution obviously also fullfils the previous problem.
        \imp{problem}: we thus can construct arbitrary many weak solutions by using the rh condition\cref{eq:rankine-hugoniot_condition} with different inermediate states.
        \todo[inline]{finish example and understand why it also fullfils the previous example}
    \end{example}
\end{examplebox}
%%% Local Variables:
%%% mode: latex
%%% TeX-master: "../../formulary"
%%% TeX-command-extra-options: "-shell-escape"
%%% End:
