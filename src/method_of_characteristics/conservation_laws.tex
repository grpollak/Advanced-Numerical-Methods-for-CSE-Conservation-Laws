\begin{sectionbox}\nospacing
   For a general introduction to the method of characteristics have a look at \Cref{subsec:method_of_characteristics}.
\end{sectionbox}
\begin{defnbox}\nospacing
    \begin{defn}[\hfill\proofref{proof:defn:characteristic_equations_for_conservation_laws}\newline Characteristic Equations for Conservation Laws]
        \label{defn:characteristic_equations_for_conservation_laws}\leavevmode\\
        A curve $\Gammac:=(\gammac(\tauc),\tauc):[0,T]\mapsto\R\times]0,T[$ in $(x,t)$-plane is a \textit{characteristic curve}
        for the conservation law \cref{eq:inviscide_conservation_law}, if:
        \begin{align}
          \dv{\tauc}\gammac(\tauc)=\flux' \left(u(\gammac(\tauc),\tauc)\right)&&0\leq\tauc\leq T\label{eq:characteristic_equations_for_conservation_laws}
        \end{align}
    \end{defn}
\end{defnbox}
\begin{corbox}\nospacing
    \begin{cor}[$u$ is constant along Characteristics]
        \begin{align}
            u(\gammac(\tauc),\tauc)=u(\gammac(0),0)=u_{0}(\gammac_{0})\label{eq:u_const_along_characteristic}
        \end{align}
    \end{cor}
\end{corbox}
\begin{propositionbox}\nospacing
    \begin{proposition}[\hfill\proofref{proof:proposition:general_solution_for_scalar_conservation_laws}\newline General Solution for scalar Conservation Laws]
        \label{proposition:general_solution_for_scalar_conservation_laws}\leavevmode\\
        The general solution of\cref{eq:inviscide_conservation_law} is given in terms of the inital condition
        maybe a non-linear equations:
        \begin{align}
          u(x,t)=u_{0}\left(x-\flux' \left(u(x,t)\right)t\right)\label{eq:general_solution_for_scalar_conservation_laws}
        \end{align}
    \end{proposition}
\end{propositionbox}
\begin{corbox}\nospacing
    \begin{cor}[\newline Riemann Problem Solution and Inital Data]
        \label{cor:riemann_problem_solution_and_scalar_conservation_laws}\leavevmode\\
        In case of a Riemann problem we see that the solution is given by a propagation of the inital data:
        \begin{align}
        u(x,t)=&\left\{\begin{aligned}
                &U_{L}&&\text{if }x-\flux'\left(u_{0}\right)t<0\\
                &U_{R}&&\text{if }x-\flux'\left(u_{0}\right)t>0
            \end{aligned}\right.\\[-1\jot]
          =&\left\{\begin{aligned}
                &U_{L}&&\text{if }x<\flux'\left(u_{0}\right)t\\
                &U_{R}&&\text{if }x>\flux'\left(u_{0}\right)t
                \end{aligned}\right.\label{eq:riemann_problem_solution_and_scalar_conservation_laws}
        \end{align}
    \end{cor}
\end{corbox}
\begin{defnbox}\nospacing
    \begin{defn}[\hfill\proofref{proof:proposition:general_solution_for_scalar_conservation_laws}
        \newline Characteristics for Sclar Conservation Laws]\label{defn:characteristics_for_sclar_conservation_laws}\leavevmode\\
        The characteristics going through $(x_{0},0)$ are given by the straight lines:
        \begin{align}
          x(t)=x_{0}+\flux'\left(u_{0}(x_{0})\right)t\label{eq:characteristics_for_sclar_conservation_laws}
        \end{align}
    \end{defn}
\end{defnbox}
\begin{sectionbox}[Problem]\nospacing
    We have seen by \cref{lemma:exploding_gradient_problem} that there exist problems where the spatial gradient explodes $\iff$ we
    have discontinuous or multivalued solutions s.t.\ \cref{eq:general_solution_for_scalar_conservation_laws} and \cref{eq:inviscide_conservation_law}
    are not even well defined.
\end{sectionbox}
%%% Local Variables:
%%% mode: latex
%%% TeX-command-extra-options: "-shell-escape"
%%% TeX-master: "../../formulary"
%%% End:
