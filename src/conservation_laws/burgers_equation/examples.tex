\begin{examplebox}\nospacing
    \begin{example}[\newline Burgers Equation Riemann Problem]\label{example:burgers_equation_riemann_problem}
        \begin{align*}
              \uldotted{u_t+u u_x}&=0 \nonumber\nalign
              u(x,0)&=u_{0}(x)=\begin{cases}
                    1&\text{if }x<0\\
                    0&\text{if }x>0
                    \end{cases}
        \end{align*}
        \begin{align*}
        &\imp{\text{ODEs}}&&\frac{\diff t}{\diff r}=1\Rightarrow \diff t=\diff r&&\frac{\diff x}{\diff r}=\ul{\dv{x}{t}}=u&&\frac{\diff u}{\diff r}=0
        \end{align*}
        \begin{align*}
          \dv{u\left(x(t),t\right)}{t}\eqs{\text{C.R.}}&=u_{t}\left(x(t),t\right)+u_{x}\left(x(t),t\right)\ul{\dv{x(t)}{t}} \\[-1\jot]
          &=\uldotted{u_{t}\left(x(t),t\right)+u_{x}\left(x(t),t\right)u}=0
        \end{align*}
        thus $u$ is constant along the \text{projectd characteristics} $x(t)=x(r)$.\\
        \imp{Problem}: let look at the inital data and the \textit{projected characteristcs}:
        \begin{align*}
          &\ul{\dv{x(t)}{t}}\Big|_{t=0}=u\left(x(t),t\right)\Big|_{t=0}=u_{0}(x_{0})=\begin{cases}1 &\text{if }x<0\\0&\text{if }x>0\end{cases}\\[-1\jot]
          &\Rightarrow\left\{\begin{aligned}
              &\int x(t)\diff x=\int 1\diff t&&\Rightarrow x(t)=x_{0}+t&&\text{if }x<0\\
              &\int\dv{x(t)}{t}\diff t=\int 0\diff t&&\Rightarrow x(t)=x_{0}&&\text{if }x>0
          \end{aligned}\right.
        \end{align*}
        \begin{figure}[H]
            \centering{
              \def\svgwidth{130pt}
              \resizebox{0.7\linewidth}{!}{\input{src/conservation_laws/burgers_equation/figures/characteristics.pdf_tex}}
            }
        \end{figure}
        Thus for $x>0$ we have intersecting \text{project.\ characteristics} i.e.\ a multivalued function that cannot be inverted.
    \end{example}
\end{examplebox}
\begin{examplebox}\nospacing
    \begin{example}[\newline Burgers Equation Continuous Initial Data]\label{example:burgers_equation_continous_initial_data}
        \begin{align*}
              \uldotted{u_t+u u_x}&=0 \nonumber\nalign
              u(x,0)&=u_{0}(x)=\begin{cases}
                    1&\text{if }x<0\\
                    1-x&\text{if }0\leq x\leq 1\\
                    0&\text{if }x>0
                    \end{cases}
        \end{align*}
        \begin{figure}[H]
            \vspace{-1em}
            \centering{
              \def\svgwidth{210pt}
                \resizebox{\linewidth}{!}{\input{src/conservation_laws/burgers_equation/figures/compression_initial_data.pdf_tex}}
            }
        \end{figure}
        \begin{figure}[H]
            \vspace{-3em}
            \centering{
              \def\svgwidth{130pt}
              \resizebox{0.7\linewidth}{!}{\input{src/conservation_laws/burgers_equation/figures/compression.pdf_tex}}
            }
        \end{figure}
        \vspace{-1em}
        Thus even for smooth initial data we will get intersection after the point $(1,1)$.
    \end{example}
\end{examplebox}
%%% Local Variables:
%%% mode: latex
%%% TeX-command-extra-options: "-shell-escape"
%%% TeX-master: "../../../formulary"
%%% End:
