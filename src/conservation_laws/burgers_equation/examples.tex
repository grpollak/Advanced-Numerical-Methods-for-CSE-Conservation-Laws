\begin{examplebox}\nospacing
    \begin{example}[\newline Burgers Equation Riemann Problem]\label{example:burgers_equation_riemann_problem}
        \begin{align*}
              \uldotted{u_t+u u_x}&=0 \nonumber \\[-1.5\jot]
              u(x,0)&=u_{0}(x)=\begin{cases}
                    1&\text{if }x<0\\
                    0&\text{if }x>0
                    \end{cases}
        \end{align*}
        \begin{flalign*}
            &\imp{\text{ODEs}}
            \qquad\left\{\qquad\begin{aligned}
                &\frac{\diff t}{\diff r}=1\Rightarrow \diff t=\ul[ulc3]{\diff r}\\
                &\frac{\diff x}{\ul[ulc3]{\diff r}}=\ul{\dv{x}{t}}=u\\
                &\frac{\diff u}{\diff r}=0
            \end{aligned}\right.&&
        \end{flalign*}
        \begin{align*}
          \dv{u\left(x(t),t\right)}{t}\eqs{\text{C.R.}}&u_{t}\left(x(t),t\right)+u_{x}\left(x(t),t\right)\ul{\dv{x(t)}{t}} \\[-1\jot]
          &=\uldotted{u_{t}\left(x(t),t\right)+u_{x}\left(x(t),t\right)u}=0
        \end{align*}
        thus $u$ is constant along the \text{projectd characteristics}
        and it holds that $u(x(t),t)=u(x(0),0)=u_{0}(x_{0})$.\\
        \\ $x(t)=x(r)$.\\
        Lets look at the inital data and the \textit{projected characteristcs}:
        \begin{align*}
          \dv{u}{r}=\dv{u}{t}=0&&\implies&&u(x,t)=\ul[ulc4]{C}
        \end{align*}
        \begin{align*}
          &\ul{\dv{x(t)}{t}}=u\left(x(t),t\right)=\ul[ulc4]{C}&&\implies&&\int_{x_0}^{x_t}\diff x(t)=\int_{0}^{t}C\diff t
        \end{align*}
        \begin{align*}
          x(t)=x_{0}+Ct=x_{0}+ut&&x_{0}=x(t)-ut
        \end{align*}
        thus we have found the general solution:
        \begin{align*}
          u(x,t)=u_{0}\left(x_{0}\right)=u_{0}\left(x-ut\right)
        \end{align*}
        now lets look at the initial conditions for $u_{0}$:
        \begin{align*}
          &\ul{\dv{x(t)}{t}}\Big|_{t=0}=u\left(x(t),t\right)\Big|_{t=0}=u_{0}(x_{0})=\begin{cases}1 &\text{if }x<0\\0&\text{if }x>0\end{cases}
        \end{align*}
        \begin{flalign*}
          &\implies&&\left\{\begin{aligned}
              &\int x(t)\diff x=\int 1\diff t&&\Rightarrow x(t)=x_{0}+t&&\text{if }x<0\\
              &\int\dv{x(t)}{t}\diff t=\int 0\diff t&&\Rightarrow x(t)=x_{0}&&\text{if }x>0
          \end{aligned}\right.&&
        \end{flalign*}
        \begin{flalign*}
          &\implies&&x(t)=
                      \left\{\begin{aligned}
                        &x_{0}+t&&\text{if }x<0\\
                        &x_{0}&&\text{if }x>0
                      \end{aligned}\right. &&
        \end{flalign*}
        \begin{figure}[H]
            \centering{
              \def\svgwidth{130pt}
              \resizebox{0.7\linewidth}{!}{\input{src/conservation_laws/burgers_equation/figures/characteristics.pdf_tex}}
            }
        \end{figure}
        For $x<0$ we obtain the solutions:
        \begin{align}
          u(x,t)=u_{0}\left(x_{0}\right)=u_{0}\left(x-ut\right)
        \end{align}
        But for $x>0$ we have intersecting \text{project.\ characteristics} i.e.\ a multivalued function that cannot be inverted
        and thus have no unique solution.
    \end{example}
\end{examplebox}
\begin{notebox}[Note]\nospacing
    The characteristic ODEs are ODEs and thus equations of one independent variable.\\
    The ODE $u(r)=u(t)$ is not to be confused with our solution $u(x,t)$.
\end{notebox}
\begin{notebox}[Physical Interpretation]
 At the singularity (\rd{shockwave}) $t=t_{\rd{crit}}$ the faster characteristic (taller part of a wave) will overtake the slower one
 (shorter part of wave), causing the wave to break.
    \begin{figure}[H]
        \centering{
            \def\svgwidth{150pt}
            \resizebox{0.8\linewidth}{!}{\input{src/method_of_characteristics/figures/waves.pdf_tex}}
        }
    \end{figure}
    \imp{Thus} their exists a physical meaning after $t_{\rd{crit}}$ where the \rd{classical solution} does not exist anymore.
    \imp{Question}: If there exists no \rd{strong solution} is there a way to find another solution? $\Rightarrow$ weak solutions.
\end{notebox}
\begin{examplebox}\nospacing
    \begin{example}[\newline Burgers Equation Continuous Initial Data]\label{example:burgers_equation_continous_initial_data}
        \begin{align*}
              \uldotted{u_t+u u_x}&=0 \nonumber\nalign
              u(x,0)&=u_{0}(x)=\begin{cases}
                    1&\text{if }x<0\\
                    1-x&\text{if }0\leq x\leq 1\\
                    0&\text{if }x>0
                    \end{cases}
        \end{align*}
        \begin{figure}[H]
            \vspace{-1em}
            \centering{
              \def\svgwidth{210pt}
                \resizebox{\linewidth}{!}{\input{src/conservation_laws/burgers_equation/figures/compression_initial_data.pdf_tex}}
            }
        \end{figure}
        \begin{figure}[H]
            \vspace{-3em}
            \centering{
              \def\svgwidth{130pt}
              \resizebox{0.7\linewidth}{!}{\input{src/conservation_laws/burgers_equation/figures/compression.pdf_tex}}
            }
        \end{figure}
        \vspace{-1em}
        Thus even for smooth initial data we will get intersection after the point $(1,1)$.
    \end{example}
\end{examplebox}
%%% Local Variables:
%%% mode: latex
%%% TeX-command-extra-options: "-shell-escape"
%%% TeX-master: "../../../formulary"
%%% End:
