\begin{sectionbox}\nospacing
    From the previous sections we have seen that the solution of conservation laws\cref{defn:conservation_law} are non-continuous s.t.\
    point values may not be well defined.
    A solution to this remedy is to work with averages, which are well defined for any integrable function and thus also for solutions of conservation laws.
\end{sectionbox}
\begin{defnbox}\nospacing
    \begin{defn}[Finite Volume Scheme Grid]\label{defn:fvs_grid}\leavevmode\\
        \imp{Space Discretization}\hfill $\left[x_{L},x_{R}\right]$
        \begin{align}
          x_{\idxj}:&=x_{L}+\left(\idxj+\frac{1}{2}\right)\Delta x&&\Delta x:=\frac{x_{R}-x_{L}}{N+1} \\[-1\jot]
          &=\frac{x_{\idxj-1/2}+x_{\idxj+1/2}}{2}
        \end{align}
        \begin{align}
          x_{\idxj\pm\frac{1}{2}}:=x_{\idxj}\pm \Delta x/2=
          \begin{cases}
              x_{L}+\idxj\Delta x &\text{-}\\
              x_{L}+(\idxj+1)\Delta x &\text{+}
          \end{cases}
        \end{align}
        \begin{flalign*}
          &&\idxj\in \left\{1,\ldots,N+1\right\}
        \end{flalign*}
        \imp{Time Discretization}\hfill$\left[0,T\right]$
        \begin{align}
          t^{n}:=n\Delta t
        \end{align}
        \vspace{-2em}
        \begin{figure}[H]
            \centering
              \def\svgwidth{180pt}
              \resizebox{0.8\linewidth}{!}{\input{src/finite_volume_methods/figures/grid.pdf_tex}}
        \end{figure}
    \end{defn}
\end{defnbox}
\begin{defnbox}\nospacing
    \begin{defn}[Control Volumes\bslash Cells]\label{defn:cells_control_volumes}\leavevmode\\
        Are the cells defined over the meshpoints $x_{\idxj}$ of the grid:
        \begin{align}
          \mathcal{C}_{\idxj}:=\left[x_{\idxj-\frac{1}{2}},x_{\idxj+\frac{1}{2}}\right)
        \end{align}
    \end{defn}
\end{defnbox}
\begin{defnbox}\nospacing
    \begin{defn}[\newline General Evolution Equation\bslash Scheme]\label{defn:general_evolution_equation}\leavevmode\\
        A $(2\idxp+1)$-point scheme is an update formula that propagates our conserved quantity of interest $U$ in time
        and relies on $(2\idxp+1)$ cell averages:
        \begin{align}
          U_{\idxj}^{\idxn+1}=H \left(U_{\idxj-\idxp}^{\idxn},\ldots,U_{\idxj+\idxp}^{\idxn}\right)
        \end{align}
    \end{defn}
\end{defnbox}
\begin{defnbox}\nospacing
    \begin{defn}[Cell Averages]\label{defn:cell_averages}\leavevmode\\
      Are averages calculated over the cells\cref{defn:cells_control_volumes} of a grid\cref{defn:fvs_grid}
      \begin{minipage}{0.45\textwidth}
      \begin{align}
        U_{\idxj}^{n}:\approx\frac{1}{\Delta x}\int\limits_{x_{\idxj-1/2}}^{x_{\idxj+1/2}}u\left(x,t^{n}\right)\diff x
      \end{align}
      \end{minipage}\hfill
      \begin{minipage}[c]{0.45\textwidth}
          \begin{figure}[H]
              \centering{
                \def\svgwidth{100pt}
                \resizebox{\linewidth}{!}{\input{src/finite_volume_methods/figures/cell_averages.pdf_tex}}
              }
          \end{figure}
      \end{minipage}
    \end{defn}
\end{defnbox}
\begin{corbox}\nospacing
    \begin{cor}[Initial Cell Averages]\label{cor:initial_cell_averages}\leavevmode\\
      Are the initial cell averages for $t=0$, given by the inital data:
      \begin{align}
        U_{\idxj}^{0}:\approx\frac{1}{\Delta x}\int_{x_{\idxj-1/2}}^{x_{\idxj+1/2}}u_{0}\left(x\right)\diff x
      \end{align}
    \end{cor}
\end{corbox}
\begin{defnbox}\nospacing
    \begin{defn}[\hfill\proofref{proof:defn:integrated_fluxes}\newline Integrated \blackrb{Boundary} Fluxes]\label{defn:integrated_fluxes}\leavevmode\\
        Is the flux of our quantity of interest $u$ over left and right boundary of the cells:
        \begin{align}
          \Flux_{\idxj\tc{seabornblue}{\pm}\frac{1}{2}}^{n}:=\int_{t_{n}}^{t_{n+1}}f \left(u\left(x_{\idxj\tc{seabornblue}{\pm}\frac{1}{2}}\right),t\right)\diff t\label{eq:integrated_fluxes}
        \end{align}
        \begin{figure}[H]
            \centering{
              \def\svgwidth{200pt}
              \resizebox{0.9\linewidth}{!}{\input{src/finite_volume_methods/figures/integrated_fluxes_continuous.pdf_tex}}
            }
        \end{figure}
    \end{defn}
\end{defnbox}
\begin{defnbox}\nospacing
    \begin{defn}[Numerical Fluxes\hfill\tcblack{$\Flux_{\idxj-1/2},\Flux_{\idxj+1/2}$}]\label{defn:numerical_fluxes}\leavevmode\\
        Are the \textit{exact} integrated or \textit{approximate} numerical fluxes across the boundaries.
    \end{defn}
\end{defnbox}
\begin{defnbox}\nospacing
    \begin{defn}[\hfill\proofref{proof:defn:finite_volume_scheme}\newline Finite Volume Scheme]\label{defn:finite_volume_scheme}\leavevmode\\
        discretize conservation laws and calculate cell averages\cref{defn:cell_averages} iteratively by integrating conservation laws\cref{defn:conservation_law}
        over the domain $\left[x_{\idxj-1/2},x_{\idxj+1/2}\right)\times \left[t^{n},t^{n+1}\right)$:
        \begin{align}
          U_{\idxj}^{n+1}&=U_{\idxj}^{n}-\frac{\Delta t}{\Delta x}\left(\Flux^{n}_{\idxj+1/2}-\Flux^{n}_{\idxj-1/2}\right)\nonumber\\[-1\jot]
          U_{\idxj}^{0}&=\frac{1}{\Delta x}\int^{x_{\idxj+1/2}}_{x_{\idxj+1/2}}U_{0}(x)\diff x\label{eq:finite_volume_scheme}
        \end{align}
    \end{defn}
\end{defnbox}
\begin{corbox}\nospacing
    \begin{cor}[\hfill\proofref{proof:cor:finite_volume_scheme_incremental_form}\newline
      Finite Volume Schemes in Incremental Form]\label{cor:finite_volume_scheme_incremental_form}
      \begin{align}
        &U_{\idxj}^{n+1}=U_{\idxj}^{n}+\mcC_{\idxj+1/2}^{n} \left(U_{\idxj+1}^{n}-U_{\idxj}^{n}\right)-\mcD_{\idxj-1/2}^{n} \left(U_{\idxj}^{n}-U_{\idxj-1}^{n}\right)\label{eq:finite_volume_scheme_incremental_form}\nonumber \\[-1\jot]
        &\text{For the FVM the coefficients are:}
      \end{align}
      \begin{align}
        \mcC_{\idxj+1/2}^{n}:=\frac{\Delta t}{\Delta x}\left(\frac{\Flux \left(u_{\idxj},u_{\idxj}\right)-\Flux \left(u_{\idxj},u_{\idxj+1}\right)}{u_{\idxj+1}-u_{\idxj}}\right)\\[-1\jot]
        \mcD_{\idxj-1/2}^{n}:=\frac{\Delta t}{\Delta x}\left(\frac{\Flux \left(u_{\idxj},u_{\idxj}\right)-\Flux \left(u_{\idxj},u_{\idxj-1}\right)}{u_{\idxj}-u_{\idxj-1}} \right)
      \end{align}
      if $\Flux$ is lipschitz\cref{defn:lipschitz_continuity} in both arguments this is equal to:
      \begin{align}
        \mcC_{\idxj+1/2}^{n}=-\frac{\Delta t}{\Delta x}\pdv{\Flux}{\mcb}\left(u_{\idxj}^{n},\cdot\right)\\[-1\jot]
        \mcD_{\idxj+1/2}^{n}=-\frac{\Delta t}{\Delta x}\pdv{\Flux}{\mca}\left(\cdot,u_{\idxj}^{n}\right)
      \end{align}
    \end{cor}
\end{corbox}
%%% Local Variables:
%%% mode: latex
%%% TeX-command-extra-options: "-shell-escape"
%%% TeX-master: "../../formulary"
%%% End:
