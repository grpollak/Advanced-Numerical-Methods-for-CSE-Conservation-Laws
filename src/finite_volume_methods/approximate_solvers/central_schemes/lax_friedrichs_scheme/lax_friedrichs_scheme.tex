\begin{defnbox}\nospacing
    \begin{defn}[Lax Friedrichs Scheme]\label{defn:lax_friedrichs_scheme}
        Chooses the wave speeds s.t.\ waves from neighboring Riemann problems do not interact with each other:
        \begin{align}
          s_{\idxj+1/2}^{l}=-\frac{\Delta x}{2\Delta t}&&s_{\idxj+1/2}^{r}=\frac{2\Delta x}{\Delta t}
        \end{align}
        with \cref{eq:central_scheme_symmetric_waves} it follows:
        \begin{align}
          \Flux_{\idxj+1/2}^{\idxn}&=\Flux^{\text{LxF}}\left(u^{\idxn}_{\idxj},u^{\idxn}_{\idxj+1}\right)\\[-1\jot]
          &=\frac{\flux\left(u_{\idxj}^{n}\right)-\flux \left(u_{\idxj+1/2}^{n}\right)}{2}-\frac{\Delta x}{2\Delta t}\left(u_{\idxj+1}^{n}-u_{\idxj}^{n}\right)\nonumber
        \end{align}
    \end{defn}
\end{defnbox}
\begin{explanationbox}\nospacing
    \begin{explanation}
        LxF makes sure that waves do not interfere with each other, that is each wave can maximally travel a distance of $\Delta x=\abs*{\frac{\Delta t}{s_{\idxj+1/2}^{l}}}$ i.e.\ to the next interface until we the next time point.
    \end{explanation}
\end{explanationbox}
\begin{sectionbox}\nospacing
    \begin{minipage}[t]{0.4\textwidth}
       \begin{proslist}
           \item Easy to implement
       \end{proslist}
    \end{minipage}
    \begin{minipage}[t]{0.55\textwidth}
       \begin{conslist}
           \item Does not take into account the local speeds
           \item Is not the most accurate
           \item Uses always an additional unnecessary grid point
       \end{conslist}
    \end{minipage}
\end{sectionbox}

%%% Local Variables:
%%% mode: latex
%%% TeX-command-extra-options: "-shell-escape"
%%% TeX-master: "../../../../../formulary"
%%% End:
