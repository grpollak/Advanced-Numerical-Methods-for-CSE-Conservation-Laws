\begin{defnbox}\nospacing
    \begin{defn}[\newline Engquist Osher Scheme]\label{defn:enquist_osher_scheme}\leavevmode\\
        Is related to\cref{defn:Rusanov_local_friedrichs_scheme} but is kind of a continuous version:
        \begin{align}
          \Flux_{\idxj+1/2}^{\idxn}=&\Flux^{\text{EO}}\left(u^{\idxn}_{\idxj},u^{\idxn}_{\idxj+1}\right)\\[-1\jot]
                                   =&\frac{\flux\left(u_{\idxj}^{n}\right)-\flux \left(u_{\idxj+1/2}^{n}\right)}{2} -\frac{1}{2}\int_{u_{\idxj}^{n}}^{u_{\idxj+1}^{n}}\abs*{f'(\thetac)}\diff\thetac\nonumber
        \end{align}
    \end{defn}
\end{defnbox}
\begin{corbox}\nospacing
    \begin{cor}[Engquist Oshner for Convex Functions]\label{cor:engquist_oshner_for_convex_functions}\leavevmode\\
        For convex functions $\flux$ with a single minimum $\alphac:=\min\flux(\thetac)$ it holds:
        \begin{align}
          \Flux^{\text{EO}}\left(u^{\idxn}_{\idxj},u^{\idxn}_{\idxj+1}\right)=\flux^{+}\left(u_{\idxj}^{\idxn}\right)+\flux^{-}\left(u_{\idxj+1}^{\idxn}\right)
        \end{align}
        \begin{align*}
            \flux^{+}\left(u\right):=\flux \left(\max \left(u,\alphac\right) \right)\\[-1\jot]
            \flux^{-}\left(u\right):=\flux \left(\min \left(u,\alphac\right) \right)
        \end{align*}
    \end{cor}
\end{corbox}
\todo[inline]{add convex cases from script}
%%% Local Variables:
%%% mode: latex
%%% TeX-command-extra-options: "-shell-escape"
%%% TeX-master: "../../../../../formulary"
%%% End:
