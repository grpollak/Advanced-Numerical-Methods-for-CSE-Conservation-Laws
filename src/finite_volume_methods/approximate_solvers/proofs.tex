\begin{proofbox}\nospacing
    \begin{proof}[Linearized Riemann Problem\cref{defn:approximate_riemann_problem}]\label{proof:defn:approximate_riemann_problem}\leavevmode\\
    \begin{align*}
      \flux \left(u\right)=\flux \left(u_{\idxj}^{\idxn}\right)+\flux' \left(\thetac^{\idxn}_{\idxj+\frac{1}{2}}\right)\left(u-u_{\idxj}^{\idxn}\right)
      \quad\thetac^{\idxn}_{\idxj+\frac{1}{2}}\in \left[u_{\idxj}^{\idxn},u_{\idxj+1}^{\idxn}\right]
    \end{align*}
    \begin{align}
      \implies&&\flux' \left(u\right)_{x}\approx\flux' \left(\thetac^{\idxn}_{\idxj+\frac{1}{2}}\right)u_{x}:=\widehat{\mcA}_{\idxj+\frac{1}{2}}u_{x}
    \end{align}
    Where $\widehat{\mcA}_{\idxj+\frac{1}{2}}\left(\thetac^{\idxn}_{\idxj+\frac{1}{2}}\right)=\flux'\left(\thetac^{\idxn}_{\idxj+\frac{1}{2}}\right)$ is a constant state around which the nonlinear flux function is linearized.\\
    The question that remains is at which point $\left(\thetac^{\idxn}_{\idxj+\frac{1}{2}}\right)\in \left[u_{\idxj}^{\idxn},u_{\idxj+1}^{\idxn}\right]$ should we evaluate
    $\widehat{\mcA}_{\idxj+\frac{1}{2}}$.
    \end{proof}
\end{proofbox}
