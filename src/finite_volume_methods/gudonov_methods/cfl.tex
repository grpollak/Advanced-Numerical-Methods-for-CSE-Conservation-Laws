\begin{defnbox}\nospacing
    \begin{defn}[CFL Condition]\label{defn:cfl_condition}\leavevmode\\
        \begin{minipage}[t]{0.5\textwidth}
            \begin{align}
              \max_{\idxj}\abs{\flux' \left(U_{\idxj}^{n}\right)}\frac{\Delta t}{\Delta x}\leq\frac{1}{2}\label{eq:cfl_condition}
            \end{align}
        \end{minipage}
        \begin{minipage}{0.45\textwidth}
            \begin{figure}[H]
                \centering{
                  \def\svgwidth{100pt}
                  \resizebox{\linewidth}{!}{\input{src/finite_volume_methods/gudonov_methods/figures/CFL.pdf_tex}}
                }
            \end{figure}
        \end{minipage}
    \end{defn}
\end{defnbox}
\begin{explanationbox}\nospacing
    \begin{explanation}
       Enforces that that neighbouring waves in a cell do not inersect each other:
        \begin{align}
          \text{\normalfont{CFL}}:=\max_{\idxj}\abs{\flux' \left(U_{\idxj}^{n}\right)}\Delta t\leq\underbrace{\frac{1}{2}\Delta x}_{\text{half the cell width}}
          \label{eq:cfl_number}
        \end{align}
    \end{explanation}
\end{explanationbox}
\begin{corbox}\nospacing
    \begin{cor}[The CFL condition can be used to calculate $\Delta t$]\label{cor:the_cfl_condition_can_be_used_to_calculate}
        \begin{align}
              \Delta t=\text{CFL}\frac{\Delta x}{\max_{\idxj}\abs{\flux' \left(U_{\idxj}^{n}\right)}}
        \end{align}
    \end{cor}
\end{corbox}
%%% Local Variables:
%%% mode: latex
%%% TeX-command-extra-options: "-shell-escape"
%%% TeX-master: "../../../formulary"
%%% End:
