\begin{proofbox}\nospacing
    \begin{proof}[Godunov Scheme\cref{defn:gudunov_riemann_scheme}]\label{proof:gudonov_scheme}
        We assume a \textit{self-similar} solution and want to have the Riemann problem at zero thus we subtract the offset $x_{\idxj+1/2}, t^{n}$:
        \begin{align}
          U_{\idxj}(x,t)=U_{\idxj}\left(\frac{x-x_{\idxj+1/2}}{t-t^{n}}\right)
        \end{align}
        Next we are only interested in the flux at the boundary $x_{\idxj+1/2}$ s.t.\ we obtain:
        \begin{align*}
          \Flux_{\idxj\tc{seabornblue}{+}\frac{1}{2}}
          &=\int_{t_{n}}^{t_{n+1}}\flux \left(u\left(x_{\idxj\tc{seabornblue}{+}\frac{1}{2}},t\right)\right)\diff t \\[-1\jot]
            &=\int_{t_{n}}^{t_{n+1}}\flux \left(U\left(\frac{x_{\idxj+1/2}-x_{\idxj+1/2}}{t-t^{n}}\right)\right)\diff t
            =\Delta t \flux \left(U\left(0\right)\right)
        \end{align*}
        where $U$ is the solution of the standard Riemann problem:
        \begin{align}
        u_{t}+\flux \left(u\right)_{x}=0
        \end{align}
        \begin{align}
            u(x,0)=\begin{cases}
                    U_{\idxj}^{n}&\text{if }x<0\\
                    U_{\idxj+1}^{n}&\text{if }x>0
                    \end{cases}
        \label{eq:riemann_problem}
        \end{align}
    \end{proof}
\end{proofbox}

%%% Local Variables:
%%% mode: latex
%%% TeX-command-extra-options: "-shell-escape"
%%% TeX-master: "../../../formulary"
%%% End:
