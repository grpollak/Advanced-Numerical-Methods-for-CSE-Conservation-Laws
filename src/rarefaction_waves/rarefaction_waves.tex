\begin{defnbox}\nospacing
    \begin{defn}[\hfill\proofref{proof:rarefaction_waves}\newline Rarefaction Wave]
        \label{defn:rarefaction_wave}\leavevmode\\
        A rarefaction wave is a \textit{self-similar} solutions of the form:
        \begin{align}
          u(x,t)=v \left(\frac{x}{t}\right)=\left(\flux^{'}\right)^{-1}\left(\frac{x}{t}\right)
        \end{align}
    \end{defn}
\end{defnbox}
\begin{corbox}\nospacing
    \begin{cor}[\hfill\exampleref{example:cor:rarefaction_solution_riemann_problem}\newline Rarefaction Solution for the Riemann Problem]\leavevmode\\
        \label{cor:rarefaction_solution_riemann_problem}
        Consider the Riemann problem \cref{eq:riemann_problem}, then a
        solution given by rarefaction wave is given by:
        \begin{align}
          u(x,t)=
          \left\{\begin{aligned}
                  &u_{L}\\
                  &\left(\flux^{'}\right)^{-1}\left(\frac{x}{t}\right)\\
                  &u_{R}
          \end{aligned}\right.&&\text{if}&&
                                    \begin{aligned}
                                     &x\leq\flux'\left(u_{L}\right)t\\
                                     \flux'\left(u_{L}\right)t<&x\leq\flux'\left(u_{R}\right)t\\
                                     &x>\flux'\left(u_{R}\right)t
                                    \end{aligned}
        \end{align}
        and satisfies the Lax-entropy condition:
    \end{cor}
\end{corbox}
%%% Local Variables:
%%% mode: latex
%%% TeX-command-extra-options: "-shell-escape"
%%% TeX-master: "../../formulary"
%%% End:
