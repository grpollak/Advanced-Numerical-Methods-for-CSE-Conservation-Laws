\begin{sectionbox}\nospacing
    We have seen that an emanating shock:
    \begin{circlelistnosep}
        \item admits infinitely many solutions
        \item does not make any sense from a physical standpoint
    \end{circlelistnosep}
    \begin{minipage}{0.5\textwidth}
    Question can we find a unique solution for the emanating shocks that does not flow out of the shocks?
    \end{minipage}\hfill
    \begin{minipage}[c]{0.45\textwidth}
      \begin{figure}[H]
          \centering{
            \def\svgwidth{120pt}
            \resizebox{\linewidth}{!}{\input{src/shock_waves/figures/emmenating_shock_general.pdf_tex}}
          }
      \end{figure}
    \end{minipage}
\end{sectionbox}
\begin{defnbox}\nospacing
    \begin{defn}[\hfill\exampleref{example:cor:rarefaction_solution_riemann_problem},\proofref{proof:rarefaction_waves}\newline Rarefaction Wave Solution for the Riemann Problem]
        \label{defn:rarefactoin_wave_solution}\leavevmode
        Let $\flux$ be a monotonic flux function s.t.:
        \begin{flalign}
            &\flux\in\sm^{2}(\R)\text{ is strictly }\left\{
            \begin{aligned}
                &\text{convex}\\
                &\text{concave}
            \end{aligned}\right.&&\text{and}&&
            \begin{aligned}
                &U_{L}<U_{R}\\
                &U_{L}>U_{R}
            \end{aligned}&&
        \end{flalign}
        Then the solution of the Riemann problem:
        \begin{align*}
        &u_{t}+\flux \left(u\right)_{x}=0
        &&u_{0}=\begin{cases}
                U_{L}&\text{if }x<0\\
                U_{R}&\text{if }x>0
                \end{cases}
        \end{align*}
        ion is given by a rarefaction wave:
        \begin{align}
          u(x,t)=
          \left\{\begin{aligned}
                  &u_{L}\\
                  &\left(\flux^{'}\right)^{-1}\left(\frac{x}{t}\right)\\
                  &u_{R}
          \end{aligned}\right.&&\text{if}&&
                                    \begin{aligned}
                                     &x\leq\min\overbrace{\left\{\flux'\left(u_{L}\right),\flux'\left(u_{R}\right)\right\}}^{:=\mca}t\\
                                     &\mca t<x\leq\mcb t\\
                                     &x>\underbrace{\max\left\{\flux'\left(u_{L}\right),\flux'\left(u_{R}\right)\right\}}_{:=\mcb}t
                                    \end{aligned}\label{eq:rarefaction_wave}
        \end{align}
        \begin{figure}[H]
            \vspace{-2em}
            \centering{
            \def\svgwidth{120pt}
            \resizebox{0.6\linewidth}{!}{\input{src/rarefaction_waves/figures/rarefaction.pdf_tex}}
            }
        \end{figure}
    \end{defn}
\end{defnbox}
\begin{corbox}\nospacing
    \begin{cor}[Lax Olenik Entropy Condition]\label{cor:lax_olenik_entropy_condition}
        \Cref{eq:rarefaction_wave} satisfies the Lax-Olenik entropy condition\cref{proposition:lax_oleinik_entropy_condition}:
        \begin{align}
          \flux'\left(u^{-}(t)\right)=s(t)=\flux'\left(u^{+}(t)\right)
        \end{align}
    \end{cor}
\end{corbox}
%%% Local Variables:
%%% mode: latex
%%% TeX-command-extra-options: "-shell-escape"
%%% TeX-master: "../../formulary"
%%% End:
