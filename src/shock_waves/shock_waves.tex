\begin{defnbox}\nospacing
    \begin{defn}[\hfill\proofref{proof:defn:shock_wave_solution}\newline Shock Wave Solution for the Riemann Problem]\label{defn:shock_wave_solution}\leavevmode\\
        For conservation laws with a \textit{monotonic} flux function $\flux$ and Riemann data \cref{defn:riemann_problem}:
        \begin{align*}
        &u_{t}+\flux \left(u\right)_{x}=0
        &&u_{0}=\begin{cases}
                U_{L}&\text{if }x<0\\
                U_{R}&\text{if }x>0
                \end{cases}
        \end{align*}
        The solution is given by
        \begin{align}
          u(x,t)=
          \begin{cases}
              U_{L}&\text{ if }x<\gammac'(t)t\\
              U_{R}&\text{ if }x>\gammac'(t)t
          \end{cases}
        \end{align}
    \end{defn}
\end{defnbox}
\begin{defnbox}\nospacing
    \begin{defn}[Types of Shocks]\label{defn:shock_types}\leavevmode\\
        There exist two types of shocks:
        \begin{circlelistnosep}
            \item \imp{Colliding Shocks}: are weak solutions where the inital data flows into the shock.
            \item \imp{Emenating Shocks}: are weak solutions where the inital data flows out of the shock.
        \end{circlelistnosep}
        \begin{minipage}[t]{0.49\textwidth}
            \vspace{-1em}
          \begin{figure}[H]
              \centering{
                \def\svgwidth{130pt}
                \resizebox{\linewidth}{!}{\input{src/shock_waves/figures/colliding_shock.pdf_tex}}
              }
              \vspace{-2em}
              \caption{\scriptsize Colliding Shocks}
          \end{figure}
        \end{minipage}
        \begin{minipage}[t]{0.49\textwidth}
            \vspace{-1em}
        \begin{figure}[H]
            \centering{
              \def\svgwidth{130pt}
              \resizebox{\linewidth}{!}{\input{src/shock_waves/figures/emmenating_shock.pdf_tex}}
            }
            \vspace{-2em}
            \caption{\scriptsize Emenating Shocks}
        \end{figure}
        \end{minipage}
    \end{defn}
\end{defnbox}


%%% Local Variables:
%%% mode: latex
%%% TeX-command-extra-options: "-shell-escape"
%%% TeX-master: "../../formulary"
%%% End:
