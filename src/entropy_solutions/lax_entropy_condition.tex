\begin{propositionbox}\nospacing
    \begin{proposition}[\hfill\blackrb{Burgers Equation}\newline Lax-Oleinik Entropy Condition]\label{proposition:lax_oleinik_entropy_condition_burger}
        Characteristics of the Burgers equation have to flow into the shock and not emanate at it:
        \begin{align}
          u^{-}(t)>s(t)>u^{+}(t)
        \end{align}
    \end{proposition}
\end{propositionbox}
\begin{propositionbox}\nospacing
    \begin{proposition}[\hfill\blackrb{Convex Functions}\newline Lax-Oleinik Entropy Condition]\label{proposition:lax_oleinik_entropy_condition}
        Characteristics of general \textit{scalar conservation law} with \textit{convex} $\flux$ should flow into the shock:
        \begin{align}
          \flux'\left(u^{-}(t)\right)>s(t)>\flux'\left(u^{+}(t)\right)\label{eq:lax_entropy_oleinik_condtion_convex}
        \end{align}
    \end{proposition}
\end{propositionbox}
\begin{explanationbox}\nospacing
    \begin{explanation}\leavevmode
        \begin{itemizenosep}
            \item For an evolution equation the flow of information should come from the initial data.
            \item We want to require that information flows into and not out from a shock.
        \end{itemizenosep}
    \end{explanation}
\end{explanationbox}

%%% Local Variables:
%%% mode: latex
%%% TeX-command-extra-options: "-shell-escape"
%%% TeX-master: "../../formulary"
%%% End:
