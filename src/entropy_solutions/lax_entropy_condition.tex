\addcontentsline{toc}{subsubsubsection}{Problems with weak solutions}
\begin{sectionbox}[Problem]\nospacing
    \textit{Emenating shocks} admit infinitely many weak solutions:
        \begin{minipage}[t]{0.49\textwidth}
        \vspace{-1em}
          \begin{figure}[H]
              \centering{
                \def\svgwidth{130pt}
                \resizebox{\linewidth}{!}{\input{src/shock_waves/figures/emmenating_shock.pdf_tex}}
              }
              \vspace{-2em}
              \caption{\scriptsize Emenating Shock 1}
          \end{figure}
        \end{minipage}
        \begin{minipage}[t]{0.49\textwidth}
        \begin{figure}[H]
        \vspace{-1em}
            \centering{
              \def\svgwidth{130pt}
              \resizebox{\linewidth}{!}{\input{src/shock_waves/figures/emmenating_shock_second.pdf_tex}}
            }
            \vspace{-2em}
            \caption{\scriptsize Emenating Shock 2}
        \end{figure}
        \end{minipage}
        Thus we allow only for \textit{colliding shocks} $\Rightarrow$ Lax-Oleinik Entropy Condition.
\end{sectionbox}
\begin{propositionbox}\nospacing
    \begin{proposition}[\hfill\blackrb{Convex Functions}\newline Lax-Oleinik Entropy Condition]\label{proposition:lax_oleinik_entropy_condition}\leavevmode\\
        The characteristics of a general \textit{scalar conservation law} with \textit{monotonic} $\flux$ function have to flow into the shock:
        \begin{align}
          \flux'\left(u^{-}(t)\right)>s(t)>\flux'\left(u^{+}(t)\right)\label{eq:lax_entropy_oleinik_condtion_convex}
        \end{align}
    \end{proposition}
\end{propositionbox}
\begin{attentionbox}
   The characteristic equations $x(t)$ are plotted in terms of the $x-t$-plane thus the axes and slopes are turned around.
\end{attentionbox}
\begin{corbox}\nospacing
    \begin{cor}[Categorization by $\flux$]
        \begin{flalign*}
            &\flux\left\{
            \begin{aligned}
                &\text{convex}  \\
                &\text{concave}
            \end{aligned}\right.&&\implies\flux'
            \left\{\begin{aligned}
                &\text{increasing}  \\
                &\text{decreasing}
            \end{aligned}\right.&&\Rightarrow\text{iff }
            \left\{\begin{aligned}
                &U_{L}>U_{R}\\
                &U_{L}<U_{R}
            \end{aligned}\right.
        \end{flalign*}
        \begin{align*}
          &\implies&&\text{physical/colliding shock}
        \end{align*}
    \end{cor}
\end{corbox}
\begin{explanationbox}\nospacing
    \begin{explanation}\leavevmode
        \begin{itemizenosep}
            \item For an evolution equation the flow of information should come from the initial data.
            \item We want to require that information flows into and not out from a shock.
        \end{itemizenosep}
    \end{explanation}
\end{explanationbox}
\begin{corbox}\nospacing
    \begin{cor}[\hfill\blackrb{Burgers Equation}\newline Lax-Oleinik Entropy Condition]
        Characteristics of the Burgers equation have to flow into the shock and not emanate at it:
        \begin{align}
          u^{-}(t)>s(t)>u^{+}(t)
        \end{align}
    \end{cor}
\end{corbox}

%%% Local Variables:
%%% mode: latex
%%% TeX-command-extra-options: "-shell-escape"
%%% TeX-master: "../../formulary"
%%% End:
