\begin{principbox}\nospacing
    \begin{princip}[Discrete Maximum Principle]\label{princip:discrete_maximum_principle}\leavevmode\\
        Is the discrete form of \cref{princip:maximum_principle}:
        \begin{align}
          \min \left(U_{\idxj-1}^{\idxn},U_{\idxj}^{\idxn},U_{\idxj+1}^{\idxn}\right)
          \leq U_{\idxj}^{\idxn+1}\leq\max\left(U_{\idxj-1}^{\idxn},U_{\idxj}^{\idxn},U_{\idxj+1}^{\idxn}\right)
        \end{align}
        \begin{figure}[H]
            \centering{
              \def\svgwidth{120pt}
              \resizebox{0.6\linewidth}{!}{\input{src/properties/discrete_maximum_principle/figures/max_disc.pdf_tex}}
            }
        \end{figure}
    \end{princip}
\end{principbox}
\begin{explanationbox}\nospacing
    \begin{explanation}
        The previous conserved quantities $U^{\idxn}$ corresponds to the initial data $U_{0}$ of the next
        Riemann problem.
    \end{explanation}
\end{explanationbox}
\begin{propertybox}\nospacing
    \begin{property}[\hfill\proofref{proof:property:consistent_monotone_three_point_schemes}
        \newline Consistent Monotone Three Point Schemes]\label{property:consistent_monotone_three_point_schemes}
        Consistent Monotone Three Point Schemes of the form:
        \begin{align}
          U_{\idxj}^{\idxn+1}=H \left(U_{\idxj-1}^{\idxn},U_{\idxj}^{\idxn},U_{\idxj}^{\idxn}\right)
        \end{align}
        satisfy the discrete maximum \cref{princip:discrete_maximum_principle}.
    \end{property}
\end{propertybox}
%%% Local Variables:
%%% mode: latex
%%% TeX-command-extra-options: "-shell-escape"
%%% TeX-master: "../../../formulary"
%%% End:
