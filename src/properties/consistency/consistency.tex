\begin{defnbox}\nospacing
    \begin{defn}[Consistent Schemes]\label{defn:consistent_schemes}\leavevmode\\
        A $2\idxp+1$ point scheme\cref{defn:general_evolution_equation} with numerical Fluxes
        \begin{align}
          \Flux_{\idxj+1/2}^{n}&=\Flux \left(U_{\idxj-\idxp+1}^{n},\ldots,U_{\idxj+\idxp}^{n}\right)\\[-1\jot]
          \Flux_{\idxj-1/2}^{n}&=\Flux \left(U_{\idxj-\idxp}^{n},\ldots,U_{\idxj+\idxp-1}^{n}\right)
        \end{align}
        is consistent if the analytical flux function $\flux$ is consistent with the numerical flux $\Flux$ that is:
        \begin{align}
          \Flux(U,\ldots,U)=\flux(u)\label{eq:consistent_schemes}
        \end{align}
    \end{defn}
\end{defnbox}
\begin{explanationbox}\nospacing
    \begin{explanation}
        This basically states that if the left and right states are consistent/have the same value then
        our approximation should do nothing and be equal to the real flux.
    \end{explanation}
\end{explanationbox}
\begin{corbox}\nospacing
    \begin{cor}[Consitency for FVM]\label{cor:consitency_for_fvm}\leavevmode\\
        A FVM\cref{defn:finite_volume_scheme} method is consistent iff for its numerical flux functions it holds that:
        \begin{align}
          \Flux(\mca,\mca)=\flux(\mca)
        \end{align}
    \end{cor}
\end{corbox}
\begin{notebox}[Note]\nospacing
    Most of the schemes that we see in the next chapter are consistent and conservative.
\end{notebox}
%%% Local Variables:
%%% mode: latex
%%% TeX-command-extra-options: "-shell-escape"
%%% TeX-master: "../../../formulary"
%%% End:
