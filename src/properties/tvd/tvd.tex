\begin{defnbox}\nospacing
    \begin{defn}[Discrete Total Variation]\label{defn:discrete_total_variation}
        Let $g$ be a function defined on $[\mca,\mcb]$ then the total variation of $g$ is given by:
        \begin{align}
          \norm{g}_{\text{TV}([\mca,\mcb])}=\sup_{\Polynomial}\sum_{\idxj=1}^{N-1}\abs*{g\left(x_{\idxj+1}\right)-g\left(x_{\idxj}\right)}
        \end{align}
        where the supremum is taken over all paritions $\Polynomial:=\left\{\mca=x_{1}<x_{2}<\cdots<x_{N}=\mcb\right\}$
    \end{defn}
\end{defnbox}
\begin{defnbox}\nospacing
    \begin{defn}[\newline Discrete Total Variation Diminishing \blackrb{TVD}]\label{defn:discrete_total_variation_diminishing}
        \begin{align}
          \norm*{U^{n+1}}_{TV(\R)}:=\sum_{\idxj}\abs*{U_{\idxj+1}^{n+1}-U_{\idxj}^{n+1}}\leq \sum_{\idxj}\abs*{U_{\idxj+1}^{n}-U_{\idxj}^{n}}\label{eq:discrete_total_variation_diminishing}
        \end{align}
    \end{defn}
\end{defnbox}
\begin{defnbox}\nospacing
    \begin{defn}[Bounded Variation]\label{defn:bounded_variation}
       \begin{align}
         \norm{g}_{\text{BV}([\mca,\mcb])}=\norm{g}_{L^1([\mca,\mcb])}+\norm{g}_{\text{TV}([\mca,\mcb])}
       \end{align}
    \end{defn}
\end{defnbox}
\begin{explanationbox}\nospacing
    \begin{explanation}
        The total variation\cref{defn:total_variation} is only a semi-norm as the TV of any constant function is zero.
        $\Rightarrow$ definition of bounded variation makes this a real norm.
    \end{explanation}
\end{explanationbox}
\begin{defnbox}\nospacing
    \begin{defn}[\newline Bounded Variation Function Space \hfill\tcblack{BV}]\label{defn:bounded_variation_functionspace}
        \begin{align}
          \text{BV}(\R):=\left\{g\in L^{1}(\R): \norm{g}_{\text{BV}(\R)}<\infty\right\}
        \end{align}
    \end{defn}
\end{defnbox}


%%% Local Variables:
%%% mode: latex
%%% TeX-command-extra-options: "-shell-escape"
%%% TeX-master: "../../../formulary"
%%% End:
